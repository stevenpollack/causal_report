\documentclass{article}
\usepackage{enumitem}
\usepackage{Sweave}
\begin{document}
\Sconcordance{concordance:Project-Proposal.tex:Project-Proposal.Rnw:%
1 2 1 1 0 64 1}




\title{\textbf{The Effect of Dieting on Sleep}}
\author{Alex Luedtke, Lucia Petito, and Steven Pollack}
\date{}
\maketitle

\section{Specify the Question}

Sleep and food consumption are two things every person thinks about each day, usually with the attitude "I should eat less and sleep more".  Researchers have conducted many studies relating obesity to sleep disorders on both extremes, and [DO LIT REVIEW]

We decided to consider how the act of trying to lose weight affects the average amount of sleep a person gets per night.  Our population of interest is healthy adults ages 25-79 who are not pregnant and have never been diagnosed by a doctor with a sleep disorder.  

We are using the NHANES data from 2009-2010 [1, 2].  Specifically, we used the following questionnaires: ``Demographic Variables and Sample Weights," ``Weight History 16 Years and Older," and ``Sleep Disorders."

Our baseline covariates are:
\begin{itemize}
\item When the interview was conducted (1 for November 1, 2009 - April 30, 2010, 2 for May 1, 2009 - October 31, 2010)
\item Subject Gender (1 for Male, 2 for Female)
\item Age in Months (300-959, or 25-79 years)
\item Race/Ethnicity (1 for Mexican American, 2 for Other Hispanic, 3 for Non-Hispanic White, 4 for Non-Hispanic Black, 5 for Other Race)
\item Education Level (1: less than 9th grade, 2: 9-11th grade (including 12th grade with no diploma), 3: high school grad/GED or equivalent, 4: some college or AA degree, 5: college graduate or above)
\item Marital Status (1: married, 2: widowed, 3: divorced, 4: separated, 5: never married, 6: living with partner)
\item Annual Household Income (12 levels, \$0-\$24,999 by \$5,000, \$25,000-\$74,999 by \$10,000,  \$75,000 - \$99,999, \$100,000 and over)
\item Body Mass Index (continuous values from ~15-50) note: this is a combination of subject weight and subject height
\end{itemize}

Because we have not learned how to deal with missing baseline covariates in class, we will only consider complete cases in this analysis.  Specifically, we deleted all people who responded ``refused" or ``don't know" or are missing data for Annual Household Income, Marital Status, Education Level, and Interview Time Period.  We deleted all people who were younger than 25 and older than 79, all pregnant women, and all people who had ever been diagnosed by a doctor with a sleep disorder (or who responded that they ``don't know").  

We have information about how much weight each person lost in the past year and whether or not that person lost the weight intentionally.  In our initial analysis we will treat whether or not a person tried to diet (see the section ``Indicators that a person is trying to diet") as the intervention variable. For all of our analyses, we will use the same outcome variable: average amount of sleep per night. However, we also have data on how much weight each individual lost in the past year (through the questions ``How much did you weigh one year ago'' and   ``How much do you weigh today'').  To capture how seriously someone is dieting, we will then perform a secondary analysis in which we discretize the amount of weight a person lost.  This will be a mediator variable, as it will fall between our intervention variable and our outcome variable.

\{Potential Sources of Bias}

We would have liked to exclude all people who are on medications that cause excessive weight gain as a side effect, such as anxiety and depression medications.  Unfortunately, our data set does not include this covariate.  

Our annual household income data 

\section{Specify the Causal Model}

\begin{align*}
W = f_W(U_W)\\
A &= f_A(W, U_A)\\
Y &= f_Y(A, W, U_Y)\\
\end{align*}

%%%% INSERT DAG HERE %%%%%%%%%%%

\section{Specify the Causal Parameter of Interest}


We are interested in the average treatment effect. Specifically, we are interested in:
$$\psi(P_{U,X}) = E_{P_{U,X}}[Y_1-Y_0]$$
Where $Y_1$ is the counterfactual average amount of sleep per night when, possibly contrary to fact, an individual is trying to diet and $Y_0$ is the counterfactual average amount of sleep per night when, possibly contrary to fact, an individual is not trying to diet.


\section{Assess Identifiability}

Condition on W
$U_A$ independent of $U_Y$
Either $U_W$ indep of $U_A$ or $U_W$ indep of $U_Y$.

\section{Commit to a Statistical Model and Target Parameter of the Observed Data Distribution}

We observe data $O = (W, A, Y) \sim P_0$ that we assume were generated by sampling 4,057 times from the data generating system specified by the SCM.

We are interested in the average treatment effect. Specifically, we are interested in:
$$\psi(P_{0}) =\mathbb{E}_W[\mathbb{E}_{P_{0}}[Y|A=1] - \mathbb{E}_{P_{0}}[Y|A=0] $$

Where $Y_1$ is the counterfactual average amount of sleep per night when, possibly contrary to fact, an individual is trying to diet and $Y_0$ is the counterfactual average amount of sleep per night when, possibly contrary to fact, an individual is not trying to diet.


\section{Estimate the Chosen Parameter of the Observed Data Distribution}

\section{Interpret Results}




%%%%%%%%%%%%%%%%%%%%%%%%%%%%
\section{Your Question in Words}

How does trying to diet affect the average amount of sleep a person gets per night?  Here, we will define ``trying to diet" as answering yes to at least one of the provided answers to the question  ``How did you try to lose weight?" on the NHANES questionnaire (the possible answers are at the end of this document).  In short, ``trying to diet" indicates an attempt within the last year to modify one's food and beverage consumption to lose weight.

Our population of interest is people ages 25-79 who are not pregnant and do not have a sleep disorder. 

\section{Your Target Causal Parameter}

We are interested in the average treatment effect. Specifically, we are interested in:
$$\psi(P_{U,X}) = E_{P_{U,X}}[Y_1-Y_0]$$
Where $Y_1$ is the counterfactual average amount of sleep per night when, possibly contrary to fact, an individual is trying to diet and $Y_0$ is the counterfactual average amount of sleep per night when, possibly contrary to fact, an individual is not trying to diet.

We will evaluate this quantity using the simple substitution estimator, IPTW, and TMLE. We will also apply marginal structural models at various stages of the analysis.

\section{A Brief Description of the Data}

We are using the NHANES data from 2009-2010 [1, 2].  Specifically, we used the following questionnaires: ``Demographic Variables and Sample Weights," ``Weight History 16 Years and Older," and ``Sleep Disorders."

Our baseline covariates are:
\begin{itemize}
\item When the interview was conducted (1 for November 1, 2009 - April 30, 2010, 2 for May 1, 2009 - October 31, 2010)
\item Subject Gender (1 for Male, 2 for Female)
\item Age in Months (300-959, or 25-79 years)
\item Race/Ethnicity (1 for Mexican American, 2 for Other Hispanic, 3 for Non-Hispanic White, 4 for Non-Hispanic Black, 5 for Other Race)
\item Education Level (1: less than 9th grade, 2: 9-11th grade (including 12th grade with no diploma), 3: high school grad/GED or equivalent, 4: some college or AA degree, 5: college graduate or above)
\item Marital Status (1: married, 2: widowed, 3: divorced, 4: separated, 5: never married, 6: living with partner)
\item Annual Household Income (12 levels, \$0-\$24,999 by \$5,000, \$25,000-\$74,999 by \$10,000,  \$75,000 - \$99,999, \$100,000 and over)
\item Body Mass Index (continuous values from ~15-50) note: this is a combination of subject weight and subject height
\end{itemize}

Because we have not learned how to deal with missing baseline covariates in class, we will only consider complete cases in this analysis.  Specifically, we deleted all people who responded ``refused" or ``don't know" or are missing data for Annual Household Income, Marital Status, Education Level, and Interview Time Period.  We deleted all people who were younger than 25 and older than 79, all pregnant women, and all people who had ever been diagnosed by a doctor with a sleep disorder (or who responded that they ``don't know").  

We have information about how much weight each person lost in the past year and whether or not that person lost the weight intentionally.  In our initial analysis we will treat whether or not a person tried to diet (see the section ``Indicators that a person is trying to diet") as the intervention variable. For all of our analyses, we will use the same outcome variable: average amount of sleep per night. However, we also have data on how much weight each individual lost in the past year (through the questions ``How much did you weigh one year ago'' and   ``How much do you weigh today'').  To capture how seriously someone is dieting, we will then perform a secondary analysis in which we discretize the amount of weight a person lost.  This will be a mediator variable, as it will fall between our intervention variable and our outcome variable.  

\section{Anticipated Challenges}

Our biggest challenge in this project will be dealing with the missing data.  For right now, we are planning to perform a complete case analysis in which we delete all people who have incomplete covariate or outcome data.

One specific covariate that will cause issues is Annual Household Income.  Because we have twelve levels, we can either consider the raw data structure or collapse it down into fewer categories. We are particularly worried about positivity violations occurring due to this covariate. Additionally, approximately 10\% of participants only provided information about whether or not their salary was above \$20,000. We have not yet decided how to treat these individuals. We are considering either omitting them from the analysis or performing some kind of imputation (will ask about this in office hours).

Because we have so many covariates, we are also concerned about positivity violations in a more general sense. When we encounter these violations, we will consider collapsing the data more, and trying to use a model that allows us to weaken our positivity assumption (IPTW).

\section{Indicators that a person is trying to diet}

Below are the potential responses to the question ``How did you try to lose weight?" [2]:

\begin{itemize}[noitemsep]
\item Ate less food
\item Switched to foods with lower calories
\item Ate less fat
\item Ate fewer carbohydrates
\item Exercised
\item Skipped meals
\item Ate ``diet'' foods or products
\item Used a liquid diet formula such as slimfast or optifast
\item Joined a weight loss program such as Weight Watchers, Jenny Craig, Tops, or Overeaters Anonymous
\item Followed a special diet such as Dr. Atkins, South Beach, other high protein or low carbohydrate diet, cabbage soup diet, Ornish, Nutrisystem, Body-for-Life
\item Took diet pills prescribed by a doctor
\item Took other pills, medicines, herbs, or supplements not needing a prescription
\item Started to smoke or began to smoke again
\item Took laxatives or vomited
\item Drank a lot of water
\item Ate more fruits, vegetables, salads
\item Ate less sugar, candy, sweets
\item Changed eating habits (didn't eat late at night, ate several small meals a day)
\item Ate less junk food or fast food
\item Other (specify)
\end{itemize}

\subsection*{}
\begin{enumerate}
	\item Centers for Disease Control and Prevention (CDC). National Center for Health Statistics (NCHS). National Health and Nutrition Examination Survey Data. Hyattsville, MD: U.S. Department of Health and Human Services, Centers for Disease Control and Prevention, 2009-2010. \\ http://www.cdc.gov/nchs/nhanes/nhanes2009-2010/nhanes09\_10.htm.
	\item Centers for Disease Control and Prevention (CDC). National Center for Health Statistics (NCHS). National Health and Nutrition Examination Survey Questionnaire (or Examination Protocol, or Laboratory Protocol). Hyattsville, MD: U.S. Department of Health and Human Services, Centers for Disease Control and Prevention, 2009-2010. \\ http://www.cdc.gov/nchs/nhanes/nhanes2009-2010/questexam09\_10.htm.
\end{enumerate}

\end{document}
